\documentclass[]{elsarticle} %review=doublespace preprint=single 5p=2 column
%%% Begin My package additions %%%%%%%%%%%%%%%%%%%
\usepackage[hyphens]{url}

  \journal{Journal of Hydrology} % Sets Journal name


\usepackage{lineno} % add
  \linenumbers % turns line numbering on
\providecommand{\tightlist}{%
  \setlength{\itemsep}{0pt}\setlength{\parskip}{0pt}}

\usepackage{graphicx}
\usepackage{booktabs} % book-quality tables
%%%%%%%%%%%%%%%% end my additions to header

\usepackage[T1]{fontenc}
\usepackage{lmodern}
\usepackage{amssymb,amsmath}
\usepackage{ifxetex,ifluatex}
\usepackage{fixltx2e} % provides \textsubscript
% use upquote if available, for straight quotes in verbatim environments
\IfFileExists{upquote.sty}{\usepackage{upquote}}{}
\ifnum 0\ifxetex 1\fi\ifluatex 1\fi=0 % if pdftex
  \usepackage[utf8]{inputenc}
\else % if luatex or xelatex
  \usepackage{fontspec}
  \ifxetex
    \usepackage{xltxtra,xunicode}
  \fi
  \defaultfontfeatures{Mapping=tex-text,Scale=MatchLowercase}
  \newcommand{\euro}{€}
\fi
% use microtype if available
\IfFileExists{microtype.sty}{\usepackage{microtype}}{}
\bibliographystyle{elsarticle-harv}
\usepackage{longtable}
\ifxetex
  \usepackage[setpagesize=false, % page size defined by xetex
              unicode=false, % unicode breaks when used with xetex
              xetex]{hyperref}
\else
  \usepackage[unicode=true]{hyperref}
\fi
\hypersetup{breaklinks=true,
            bookmarks=true,
            pdfauthor={},
            pdftitle={Do larger catchments respond different to forest cover change? Re-analysing a global data set.},
            colorlinks=false,
            urlcolor=blue,
            linkcolor=magenta,
            pdfborder={0 0 0}}
\urlstyle{same}  % don't use monospace font for urls

\setcounter{secnumdepth}{0}
% Pandoc toggle for numbering sections (defaults to be off)
\setcounter{secnumdepth}{0}

% Pandoc citation processing

% Pandoc header



\begin{document}
\begin{frontmatter}

  \title{Do larger catchments respond different to forest cover change?
Re-analysing a global data set.}
    \author[The University of Sydney, INIA]{R. Willem Vervoort\corref{1}}
   \ead{willem.vervoort@sydney.edu.au} 
    \author[INIA]{Eliana Nervi}
   \ead{eliananervi@gmail.com} 
    \author[IMFIA]{Jimena Alonso\corref{2}}
   \ead{jalonso@fing.edu.uy} 
      \address[The University of Sydney]{School of Life and Environmental Sciences, The University of Sydney,
Sydney, NSW 2006, Australia}
    \address[INIA]{INIA, Uruguay}
    \address[IMFIA]{Institute of Fluid Mechanics and Environmental Engineering, School of
Engineering, Universidad de la República, 11200 Montevideo, Departamento
de Montevideo, Uruguay}
      \cortext[1]{Corresponding Author}
    \cortext[2]{Equal contribution}
  
  \begin{abstract}
  This is the abstract.
  
  It consists of two paragraphs.
  \end{abstract}
  
 \end{frontmatter}

\hypertarget{introduction}{%
\section{Introduction}\label{introduction}}

\hypertarget{introduction-1}{%
\subsection{Introduction}\label{introduction-1}}

There has been an long and on-going discussion in the hydrologcal
literature around the impact of forests on streamflow (Andréassian,
2004; Brown et al., 2013, 2005; Jackson et al., 2005; Zhang et al.,
2017). The historic work highlights provides a general consensus that if
forest areas increase, streamflow decreases and vice-versa. The most
dramatic result in relation to this, is Figure 5 in Zhang et al. (2011)
indicating (or Australian watersheds) a 100\% decrease in stream flow
for watersheds with 100\% forest cover. However, on the other end of the
spectrum, in a series of French watersheds (Cosandey et al., 2005),
there was no change in streamflow characteristics in 2 of the three
watersheds studied in relation to deforestation.

There have been several review papers aiming to summarize different
studies across the globe, in relation to paired watershed studies (Bosch
and Hewlett, 1982; Brown et al., 2005) and more generally (Jackson et
al., 2005; Zhang et al., 2017). These studies are aiming to generalize
the individual findings and to identify if there are global trends or
relationships that can be developed. The most recent review (Zhang et
al., 2017) developed an impressive database of watershed studies in
relation to changes in streamflow due to changes in forest cover based
on a global data set. This dataset, which covers over 250 studies are
described in terms of the change in streamflow as a result of the change
in forest cover, where studies related to both forestation (increase in
forest cover) and deforestation (decrease in forest cover) were
included.

The conclusions of the paper (Zhang et al., 2017) suggest that there is
a distinct difference in the change in flow as a result of forestation
or deforestation between small watersheds, defined as \textless{} 1000
km\textsuperscript{2} and large watersheds \textgreater{} 1000
km\textsuperscript{2}. While for small watersheds there was no real
change in runoff with changes in cover, for large watersheds there was a
clear trend showing a decrease in runoff with and increase in forest
cover. Their main conclusion was that the response in annual runoff to
forest cover was scale dependent and appeared to be more sensitive to
forest cover change in water limited watersheds relative to energy
limited watershed (Zhang et al., 2017).

Encouraged by the work presented by Zhang et al. (2017) and the
fantastic database of studies presented by these authors, we believe we
can add to the discussion by presenting further analysis of the data and
by adding further watersheds and enhancements to the data base.

In particular, the main method in the work by Zhang et al. (2017) is
using simple linear regression. And the main assumption is that the
threshold at 1000 km\textsuperscript{2} is a distinct separation between
``small'' and ``large'' watersheds. Given the fantastic data set
collected, the analysis can be easily expanded to look at interactions
between the terms and to test the assumption of a distinct threshold at
1000 km\textsuperscript{2}.

In particular, the objective of this paper is to 1) enhance the data set
from Zhang et al. (2017) with further watersheds and spatial coordinates
and 2) to analyse the possibility of non-linear and partial effects of
the different factors and variables in the data base using generalised
linear (GLM) and generalised additive models (GAM Wood (2006)). Finally
we hope to point to further research that can expand our work and that
outlined Zhang et al. (2017) to better understand the impact of forest
cover change on streamflow.

\hypertarget{methods}{%
\section{Methods}\label{methods}}

\hypertarget{the-original-data-set}{%
\subsection{The original data set}\label{the-original-data-set}}

The starting point of this paper is the data base of studies which were
included in Zhang et al. (2017) as supplementary material. The columns
in this data set are the watershed number, the watershed name, the Area
in km\textsuperscript{2}, the annual average precipitation (Pa) in mm,
the forest type, hydrological regime, and climate type, the change in
forest cover in \% (\(\Delta\)F\%) and the change in streamflow in \%
(\(\Delta\)Qf(\%), based on equation 1 in Zhang et al. (2017)), the
precipitation data type, the assessment technique, and the source of the
info, which is a citation. Several of these columns contain
abbreviations to describe the different variables, which are summarised
in Table 1.

Table 1 Summary of abbreviations of factors used in the Zhang et al.
(2017) data set

\begin{longtable}[]{@{}lll@{}}
\toprule
Factor & Abbreviation & Definition\tabularnewline
\midrule
\endhead
forest type & CF & coniferous forest\tabularnewline
& BF & broadleaf forest\tabularnewline
& MF & mixed forest\tabularnewline
hydrological regime & RD & rain dominated\tabularnewline
& SD & snow dominated\tabularnewline
climate type & EL & energy limited\tabularnewline
& WL & water limited\tabularnewline
& EQ & equitant\tabularnewline
precipitation data type & OB & observed\tabularnewline
& SG & spatial gridded\tabularnewline
& MD & modelled\tabularnewline
assessment technique & PWE & paired watershed experiment\tabularnewline
& QPW & quasi-paired watershed experiment\tabularnewline
& HM & hydrological modelling\tabularnewline
& EA & elastictity analysis\tabularnewline
& SH & combined use of statistical methods\tabularnewline
& & and hydrographs\tabularnewline
\bottomrule
\end{longtable}

While the Zhang et al. (2017) use the dryness index in their analysis,
potential or reference evapotranspiration is not part of the data set.
We combined the tables for small (\textless{} 1000
km\textsuperscript{2}) and large (\textgreater= 1000
km\textsuperscript{2}) watreshed data sets in our analysis. Some small
naming errors and citations for some of the data sets for some of the
small watersheds were fixed as we were familiar with the studies. But
overall the original data set was not changed.

\hypertarget{additional-data-collection}{%
\subsection{Additional data
collection}\label{additional-data-collection}}

To enhance the existing data set, this study added additional variables.
The first variables added were the latitude and longitude for the center
of the watershed as an approximation of its spatial location. Using this
information annual average reference evapotranspiration (E0) was
extracted from \textbf{XXXXX} if value of E0 were not available from the
original papers. For large water watersheds, this value, similar to
annual average rainfall, is an approximation of the climate at the
location.

The length of the study can be a variable influencing the change in flow
(e.g.~Jackson et al., 2005) and therefore, the length of the different
studies was extracted from the references provided by Zhang et al.
(2017).

Several additional data points from watershed studies were extracted
from Zhang et al. (2011), Zhao et al. (2010), Borg et al. (1988),
Thornton et al. (2007), Zhou et al. (2010), Rodriguez et al. (2010),
Ruprecht et al. (1991) and Peña-Arancibia et al. (2012), and these were
checked against the existing studies to prevent overlap. In the citation
column in the data set, in general the main reference for the calculated
change in streamflow was used, because sometimes the original study did
not provide the quantification of the change in streamflow (i.e.~Table 6
in Zhang et al. (2011))

The final column in the improved data set is a ``notes'' column, which
is not further used in the analysis, but gives context to some of the
data for future research.

\hypertarget{statistical-modelling}{%
\subsection{Statistical modelling}\label{statistical-modelling}}

To estimate how the change in streamflow is affected by the change in
forest cover while considering the effects of the other variables, we
applied generalised additive modelling (GAM) (Wood, 2006). In particular
we fitted the following initial model to replicate the variables in the
analysis from Zhang et al. (2017):

\[\tag{1}
\begin{aligned}
\Delta \%Q \sim ~&\Delta \%forest + s(Pa) + s(Area) +  {forest~type} + \\  &{climate~type} + {assessment~type} + {hydrologic~regime} + \varepsilon
\end{aligned}\]

However, the overall skewed distribution of the predictant
(\(\Delta \%Q\)) is problematic, and this results in a skewed
distribution of the GAM model residuals. As a result we transformed
\(\Delta \%Q\) and \(\Delta \%forest\) back to fractions (0 - 1) and log
transformed using \(log10(x + 1)\), where \(x\) is either \(\Delta Q\)
or \(\Delta forest\). This means that the model residuals are
\(\sim N(0,\sigma^2)\) results in the following equation:

\[\tag{2}
\begin{aligned}
log10(\Delta Q) \sim ~&log10(\Delta forest) + s(Pa, k = 3) + s(Area, k = 3) +  {forest~type} + \\  &{climate~type} + {assessment~type} + {hydrologic~regime} + \varepsilon
\end{aligned}\]

In this model, the assumption is that all continuous variables (such as
Pa) can have a linear or non-linear relationship with
\(log10(\Delta Q)\). This means that a smooth function \(s()\) is
applied to the variable. To restrict the smoothness of the fit, the
smoothness factor \(k\) is restricted to a value of 3 (Wood, 2006). This
restriction was applied to smooth variables throughout this paper and we
have dropped this from the subsequent equations.

For the model in equation 2, we only used the data from Zhang et al.
(2017) to make sure that the additional watersheds added to the data set
did not influence the analysis. Given that in Zhang et al. (2017),
dryness (\(\frac{E0}{Pa}\)) is used to look at variations in the change
in flow, we also fitted the following model:

\[\tag{3}
\begin{aligned}
log10(\Delta Q) \sim ~&log10(\Delta forest) + s(\frac{E0}{Pa}) + s(Area) +  {forest~type} + \\  &{climate~type} + {assessment~type} + {hydrologic~regime} + \varepsilon
\end{aligned}\]

Subsequently, using the full data set, including the additional
watersheds and the additional variables the following two models were
fitted:

\[\tag{4}
\begin{aligned}
log10(\Delta Q) \sim ~&log10(\Delta forest) + s(Pa) + s(Area) + s(Latitude) + s(Longitude) + \\
& s(years) + {forest~type} + {climate~type} + {assessment~type} + \\ & {hydrologic~regime} + \varepsilon
\end{aligned}\]

\[\tag{5}
\begin{aligned}
log10(\Delta Q) \sim ~&log10(\Delta forest) + s(\frac{E0}{Pa}) + s(Area) + s(Latitude) + s(Longitude) + \\
& s(years) + {forest~type} + {climate~type} + {assessment~type} + \\ & {hydrologic~regime} + \varepsilon
\end{aligned}\]

The results were analysed to identify:\\
1. the significance of the different variables\\
2. the direction of the categorical or shape of the smooth variables

\hypertarget{references}{%
\section*{References}\label{references}}
\addcontentsline{toc}{section}{References}

\hypertarget{refs}{}
\leavevmode\hypertarget{ref-andreassian2004}{}%
Andréassian, V., 2004. Waters and forests: From historical controversy
to scientific debate. Journal of Hydrology 291, 1--27.
doi:\href{https://doi.org/https://doi.org/10.1016/j.jhydrol.2003.12.015}{https://doi.org/10.1016/j.jhydrol.2003.12.015}

\leavevmode\hypertarget{ref-borg1988}{}%
Borg, H., Bell, R.W., Loh, I.C., 1988. Streamflow and stream salinity in
a small water supply catchment in southwest western australia after
reforestation. Journal of Hydrology 103, 323--333.
doi:\href{https://doi.org/https://doi.org/10.1016/0022-1694(88)90141-2}{https://doi.org/10.1016/0022-1694(88)90141-2}

\leavevmode\hypertarget{ref-hewlett1984}{}%
Bosch, J.M., Hewlett, J.D., 1982. A review of catchment experiments to
determine the effect of vegetation changes on water yield and
evapotranspiration. Journal of Hydrology 55, 3--23.

\leavevmode\hypertarget{ref-brown2013}{}%
Brown, A.E., Western, A.W., McMahon, T.A., Zhang, L., 2013. Impact of
forest cover changes on annual streamflow and flow duration curves.
Journal of Hydrology 483, 39--50.
doi:\href{https://doi.org/http://dx.doi.org/10.1016/j.jhydrol.2012.12.031}{http://dx.doi.org/10.1016/j.jhydrol.2012.12.031}

\leavevmode\hypertarget{ref-brown2005}{}%
Brown, A.E., Zhang, L., McMahon, T.A., Western, A.W., Vertessy, R.A.,
2005. A review of paired catchment studies for determining changes in
water yield resulting from alterations in vegetation. Journal of
Hydrology 310, 28--61.

\leavevmode\hypertarget{ref-cosandey2005}{}%
Cosandey, C., Andréassian, V., Martin, C., Didon-Lescot, J.F., Lavabre,
J., Folton, N., Mathys, N., Richard, D., 2005. The hydrological impact
of the mediterranean forest: A review of french research. Journal of
Hydrology 301, 235--249.
doi:\href{https://doi.org/https://doi.org/10.1016/j.jhydrol.2004.06.040}{https://doi.org/10.1016/j.jhydrol.2004.06.040}

\leavevmode\hypertarget{ref-jackson2005}{}%
Jackson, R.B., Jobbagy, E.G., Avissar, R., Roy, S.B., Barrett, D.J.,
Cook, C.W., Farley, K.A., Maitre, D.C. le, McCarl, B.A., Murray, B.C.,
2005. Trading water for carbon with biological carbon sequestration.
Science 310, 1944--1947.
doi:\href{https://doi.org/10.1126/science.1119282}{10.1126/science.1119282}

\leavevmode\hypertarget{ref-pena-arancibia2012}{}%
Peña-Arancibia, J.L., Dijk, A.I.J.M. van, Guerschman, J.P., Mulligan,
M., Bruijnzeel, L.A., McVicar, T.R., 2012. Detecting changes in
streamflow after partial woodland clearing in two large catchments in
the seasonal tropics. Journal of Hydrology 416-417, 60--71.
doi:\href{https://doi.org/https://doi.org/10.1016/j.jhydrol.2011.11.036}{https://doi.org/10.1016/j.jhydrol.2011.11.036}

\leavevmode\hypertarget{ref-rodriguez2010}{}%
Rodriguez, D.A., Tomasella, J., Linhares, C., 2010. Is the forest
conversion to pasture affecting the hydrological response of amazonian
catchments? Signals in the ji-paraná basin. Hydrological Processes 24,
1254--1269.
doi:\href{https://doi.org/https://doi.org/10.1002/hyp.7586}{https://doi.org/10.1002/hyp.7586}

\leavevmode\hypertarget{ref-ruprechtetal1991}{}%
Ruprecht, J.K., Schofield, N.J., Crombie, D.S., Vertessy, R.A.,
Stoneman, G.L., 1991. Early hydrological response to intense forest
thinning in southwestern australia. Journal of Hydrology 127, 261--277.
doi:\href{https://doi.org/https://doi.org/10.1016/0022-1694(91)90118-2}{https://doi.org/10.1016/0022-1694(91)90118-2}

\leavevmode\hypertarget{ref-thornton2007}{}%
Thornton, C.M., Cowie, B.A., Freebairn, D.M., Playford, C.L., 2007. The
brigalow catchment study: II*. Clearing brigalow (acacia harpophylla)
for cropping or pasture increases runoff. Australian Journal of Soil
Research 45, 496--511.
doi:\href{https://doi.org/doi:10.1071/SR07064}{doi:10.1071/SR07064}

\leavevmode\hypertarget{ref-wood2006}{}%
Wood, S., 2006. Generalized additive models: An introduction with r. CRC
Press, Boca Raton, FL.

\leavevmode\hypertarget{ref-zhang2011}{}%
Zhang, L., Zhao, F., Chen, Y., Dixon, R.N.M., 2011. Estimating effects
of plantation expansion and climate variability on streamflow for
catchments in australia. Water Resources Research 47, W12539.
doi:\href{https://doi.org/10.1029/2011wr010711}{10.1029/2011wr010711}

\leavevmode\hypertarget{ref-zhang2017}{}%
Zhang, M., Liu, N., Harper, R., Li, Q., Liu, K., Wei, X., Ning, D., Hou,
Y., Liu, S., 2017. A global review on hydrological responses to forest
change across multiple spatial scales: Importance of scale, climate,
forest type and hydrological regime. Journal of Hydrology 546, 44--59.
doi:\href{https://doi.org/https://doi.org/10.1016/j.jhydrol.2016.12.040}{https://doi.org/10.1016/j.jhydrol.2016.12.040}

\leavevmode\hypertarget{ref-zhao2010}{}%
Zhao, F., Zhang, L., Xu, Z., Scott, D.F., 2010. Evaluation of methods
for estimating the effects of vegetation change and climate variability
on streamflow. Water Resources Research 46, W03505.
doi:\href{https://doi.org/10.1029/2009wr007702}{10.1029/2009wr007702}

\leavevmode\hypertarget{ref-zhou2010}{}%
Zhou, G., Wei, X., Luo, Y., Zhang, M., Li, Y., Qiao, Y., Liu, H., Wang,
C., 2010. Forest recovery and river discharge at the regional scale of
guangdong province, china. Water Resources Research 46.
doi:\href{https://doi.org/https://doi.org/10.1029/2009WR008829}{https://doi.org/10.1029/2009WR008829}


\end{document}


