\documentclass[]{elsarticle} %review=doublespace preprint=single 5p=2 column
%%% Begin My package additions %%%%%%%%%%%%%%%%%%%
\usepackage[hyphens]{url}

  \journal{Journal of Hydrology} % Sets Journal name


\usepackage{lineno} % add
  \linenumbers % turns line numbering on
\providecommand{\tightlist}{%
  \setlength{\itemsep}{0pt}\setlength{\parskip}{0pt}}

\usepackage{graphicx}
\usepackage{booktabs} % book-quality tables
%%%%%%%%%%%%%%%% end my additions to header

\usepackage[T1]{fontenc}
\usepackage{lmodern}
\usepackage{amssymb,amsmath}
\usepackage{ifxetex,ifluatex}
\usepackage{fixltx2e} % provides \textsubscript
% use upquote if available, for straight quotes in verbatim environments
\IfFileExists{upquote.sty}{\usepackage{upquote}}{}
\ifnum 0\ifxetex 1\fi\ifluatex 1\fi=0 % if pdftex
  \usepackage[utf8]{inputenc}
\else % if luatex or xelatex
  \usepackage{fontspec}
  \ifxetex
    \usepackage{xltxtra,xunicode}
  \fi
  \defaultfontfeatures{Mapping=tex-text,Scale=MatchLowercase}
  \newcommand{\euro}{€}
\fi
% use microtype if available
\IfFileExists{microtype.sty}{\usepackage{microtype}}{}
\bibliographystyle{elsarticle-harv}
\ifxetex
  \usepackage[setpagesize=false, % page size defined by xetex
              unicode=false, % unicode breaks when used with xetex
              xetex]{hyperref}
\else
  \usepackage[unicode=true]{hyperref}
\fi
\hypersetup{breaklinks=true,
            bookmarks=true,
            pdfauthor={},
            pdftitle={Short Paper},
            colorlinks=false,
            urlcolor=blue,
            linkcolor=magenta,
            pdfborder={0 0 0}}
\urlstyle{same}  % don't use monospace font for urls

\setcounter{secnumdepth}{0}
% Pandoc toggle for numbering sections (defaults to be off)
\setcounter{secnumdepth}{0}

% Pandoc citation processing

% Pandoc header



\begin{document}
\begin{frontmatter}

  \title{Short Paper}
    \author[The University of Sydney, INIA]{R. Willem Vervoort\corref{1}}
   \ead{willem.vervoort@sydney.edu.au} 
    \author[INIA]{Eliana Nervi}
   \ead{eliananervi@gmail.com} 
    \author[IMFIA]{Jimena Alonso\corref{2}}
   \ead{jalonso@fing.edu.uy} 
      \address[The University of Sydney]{School of Life and Environmental Sciences, The University of Sydney,
Sydney, NSW 2006, Australia}
    \address[INIA]{INIA, Uruguay}
    \address[IMFIA]{Institute of Fluid Mechanics and Environmental Engineering, School of
Engineering, Universidad de la República, 11200 Montevideo, Departamento
de Montevideo, Uruguay}
      \cortext[1]{Corresponding Author}
    \cortext[2]{Equal contribution}
  
  \begin{abstract}
  This is the abstract.
  
  It consists of two paragraphs.
  \end{abstract}
  
 \end{frontmatter}

\emph{Text based on elsarticle sample manuscript, see
\url{http://www.elsevier.com/author-schemas/latex-instructions\#elsarticle}}

\hypertarget{introduction}{%
\section{Introduction}\label{introduction}}

\hypertarget{introduction-1}{%
\subsection{Introduction}\label{introduction-1}}

There has been an long and on-going discussion in the hydrologcal
literature around the impact of forests on streamflow (Andréassian,
2004; Brown et al., 2013, 2005; Jackson et al., 2005; Zhang et al.,
2017). The historic work highlights provides a general consensus that if
forest areas increase, streamflow decreases and vice-versa. The most
dramatic result in relation to this, is Figure 5 in Zhang et al. (2011)
indicating (or Australian watersheds) a 100\% decrease in stream flow
for watersheds with 100\% forest cover. However, on the other end of the
spectrum, in a series of French watersheds (Cosandey et al., 2005),
there was no change in streamflow characteristics in 2 of the three
watersheds studied in relation to deforestation.

There have been several review papers aiming to summarize different
studies across the globe, in relation to paired watershed studies (Bosch
and Hewlett, 1982; Brown et al., 2005) and more generally (Jackson et
al., 2005; Zhang et al., 2017). These studies are aiming to generalize
the individual findings and to identify if there are global trends or
relationships that can be developed. The most recent review (Zhang et
al., 2017) developed an impressive database of watershed studies in
relation to changes in streamflow due to changes in forest cover based
on a global data set. This dataset, that covers over 250 studies are
described in terms of the change in streamflow as a result of the change
in forest cover, where studies related to both forestation (increase in
forest cover) and deforestation (decrease in forest cover) were
included.

The conclusions of the paper (Zhang et al., 2017) suggest that there is
a distinct difference in the change in flow as a result of forestation
or deforestation between small watersheds, defined as \textless{} 1000
km\textsuperscript{2} and large watersheds \textgreater{} 1000
km\textsuperscript{2}. While for small watersheds there was no real
change in runoff with changes in cover, for large watersheds there was a
clear trend showing a decrease in runoff with and increase in forest
cover. Their main conclusion was that the response in annual runoff to
forest cover was scale dependent and appeared to be more sensitive to
forest cover change in water limited watersheds relative to energy
limited watershed (Zhang et al., 2017).

Encouraged by the work presented by Zhang et al. (2017) and the
fantastic database of studies presented by these authors, we believe we
can add to the discussion by presenting further analysis of the data and
by adding further watersheds and enhancements to the data base.

In particular, the main method in the work by Zhang et al. (2017) is
using simple linear regression. And the main assumption is that the
threshold at 1000 km\textsuperscript{2} is a distinct separation between
``small'' and ``large'' watersheds. Given the fantastic data set
collected, the analysis can be easily expanded to look at interactions
between the terms and to test the assumption of a distinct threshold at
1000 km\textsuperscript{2}.

In particular, the objective of this paper is to 1) enhance the data set
from ({\textbf{???}}) with further watersheds and spatial coordinates
and 2) to analyse the possibility of non-linear and partial effects of
the different factors and variables in the data base using generalised
linear (GLM) and generalised additive models (GAM Wood (2006)). Finally
we hope to point to further research that can expand our work and that
outlined Zhang et al. (2017) to better understand the impact of forest
cover change on streamflow.

\hypertarget{front-matter}{%
\section{Front matter}\label{front-matter}}

The author names and affiliations could be formatted in two ways:

\begin{enumerate}
\def\labelenumi{(\arabic{enumi})}
\item
  Group the authors per affiliation.
\item
  Use footnotes to indicate the affiliations.
\end{enumerate}

See the front matter of this document for examples. You are recommended
to conform your choice to the journal you are submitting to.

\hypertarget{references}{%
\section*{References}\label{references}}
\addcontentsline{toc}{section}{References}

\hypertarget{refs}{}
\leavevmode\hypertarget{ref-andreassian2004}{}%
Andréassian, V., 2004. Waters and forests: From historical controversy
to scientific debate. Journal of Hydrology 291, 1--27.
doi:\href{https://doi.org/https://doi.org/10.1016/j.jhydrol.2003.12.015}{https://doi.org/10.1016/j.jhydrol.2003.12.015}

\leavevmode\hypertarget{ref-hewlett1984}{}%
Bosch, J.M., Hewlett, J.D., 1982. A review of catchment experiments to
determine the effect of vegetation changes on water yield and
evapotranspiration. Journal of Hydrology 55, 3--23.

\leavevmode\hypertarget{ref-brown2013}{}%
Brown, A.E., Western, A.W., McMahon, T.A., Zhang, L., 2013. Impact of
forest cover changes on annual streamflow and flow duration curves.
Journal of Hydrology 483, 39--50.
doi:\href{https://doi.org/http://dx.doi.org/10.1016/j.jhydrol.2012.12.031}{http://dx.doi.org/10.1016/j.jhydrol.2012.12.031}

\leavevmode\hypertarget{ref-brown2005}{}%
Brown, A.E., Zhang, L., McMahon, T.A., Western, A.W., Vertessy, R.A.,
2005. A review of paired catchment studies for determining changes in
water yield resulting from alterations in vegetation. Journal of
Hydrology 310, 28--61.

\leavevmode\hypertarget{ref-cosandey2005}{}%
Cosandey, C., Andréassian, V., Martin, C., Didon-Lescot, J.F., Lavabre,
J., Folton, N., Mathys, N., Richard, D., 2005. The hydrological impact
of the mediterranean forest: A review of french research. Journal of
Hydrology 301, 235--249.
doi:\href{https://doi.org/https://doi.org/10.1016/j.jhydrol.2004.06.040}{https://doi.org/10.1016/j.jhydrol.2004.06.040}

\leavevmode\hypertarget{ref-jackson2005}{}%
Jackson, R.B., Jobbagy, E.G., Avissar, R., Roy, S.B., Barrett, D.J.,
Cook, C.W., Farley, K.A., Maitre, D.C. le, McCarl, B.A., Murray, B.C.,
2005. Trading water for carbon with biological carbon sequestration.
Science 310, 1944--1947.
doi:\href{https://doi.org/10.1126/science.1119282}{10.1126/science.1119282}

\leavevmode\hypertarget{ref-wood2006}{}%
Wood, S., 2006. Generalized additive models: An introduction with r. CRC
Press, Boca Raton, FL.

\leavevmode\hypertarget{ref-zhang2011}{}%
Zhang, L., Zhao, F., Chen, Y., Dixon, R.N.M., 2011. Estimating effects
of plantation expansion and climate variability on streamflow for
catchments in australia. Water Resources Research 47, W12539.
doi:\href{https://doi.org/10.1029/2011wr010711}{10.1029/2011wr010711}

\leavevmode\hypertarget{ref-zhang2017}{}%
Zhang, M., Liu, N., Harper, R., Li, Q., Liu, K., Wei, X., Ning, D., Hou,
Y., Liu, S., 2017. A global review on hydrological responses to forest
change across multiple spatial scales: Importance of scale, climate,
forest type and hydrological regime. Journal of Hydrology 546, 44--59.
doi:\href{https://doi.org/https://doi.org/10.1016/j.jhydrol.2016.12.040}{https://doi.org/10.1016/j.jhydrol.2016.12.040}


\end{document}


